\documentclass{article}
\usepackage{inputenc}
\usepackage{ifpdf}
\usepackage{mla}
\begin{document}
\begin{mla}{John Wesley}{Hayhurst}{Dr. Wrayno}{Math 301}{\today}{Major Field Autobiography}
Ever since I was little I have had an interest in computers. It sounds corny but the earliest memory that I can recall is sitting in front of my grandparent's computer and wondering just how the magical mystery box in front of me was working. When I was ten years old I started to develop an interest in math as most people were struggling with concepts as what is $x$, I was very quickly able to grasp the concept of a variable. Fast forward about twelve years and I am about to finish my degree as well as starting grad classes early for Computer Science. In addition I am also finishing up my minor in Mathematics and pursuing a major during my graduate studies. I have had an interest in computers and mathematics for a while and the story of how I got here, where I am today, is an interesting one to say the least.\\

Let us start off with my real passion and drive, and the main reason I am in college, computers. As I have stated before my interest in computers started with my grandparents. My grandparents were the first in my immediate family to own a computer. I remember so vividly sitting in front of that monitor and instantly thinking that this is a TV that I can control. I was instantly hooked and wanted to learn everything about the magical mystery TV. With a little bit of research and understanding I then found out that the magical mystery TV was not a TV, but more of what people think computers are today. Fast forward a couple of years and then at a cub scout meeting we took apart a computer, we took away the proverbial curtain and I then learned that a computer was nothing more than transistors and silicon. This drove me to find out even more about how computers worked and functioned. My parents took note of how fascinated I was with computers and sent me to a summer camp where I learned the basics of web programming and programming simple games with visual basic. Then with a passion for computers I went off to college to learn and absorb as much information as possible. Now I am here at Christopher Newport University getting my degree in Computer Science and currently starting my graduate classes for a Masters Degree.\\

Mathematics was something that I was good at. I always got a joy and satisfaction from helping my peers and as the school "math wiz" I was able to reinforce my mathematics skills. When I was in elementary school the first time I knew I was good at math was the timed multiplication test. I was able to do them pretty quick and graded top in my class for math. My teacher at the time took note of how quick I was able to solve these and suggested I played the math card game 24. 24 was a game with four numbers on a card, the goal of the game was to be the first to use addition, subtraction, multiplication and division to get 24. At my school I was really good at this game, I was so good in fact that I went to a state wide "math tournament" to play this game against other school districts. Then move forward to middle school and high school and algebra came fairly simple and calculus was nothing different. The maths that were being taught to me in high school and middle school came easy and were not challenging me. To supplement this need of a challenge I started to teach myself more advanced maths that would not be brought up until I was in college. Know that I am in college I keep challenging myself and try to push the limits of what I know about mathematics. This year I am finishing up my Mathematics minor and will continue to pursue a mathematics major during my graduate studies.\\

Knowledge to me is something to strive for, as well as admitting that you may have a lack of knowledge as well. From a young age I was always curious about how things worked and finding a solution to the unknown. Computer science helped me pursue the understand of how things work and mathematics helped me pursue finding a solution to the unknown. I do not know where the future will take me but I know that I will continue to pursue the unknown in the pursuit of knowledge.
\end {mla}
\end{document}
