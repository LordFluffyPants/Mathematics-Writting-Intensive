\documentclass[12pt]{article}
\usepackage{inputenc}
\usepackage{ifpdf}
\usepackage{mla}
\begin{document}
\begin{mla}{John Wesley}{Hayhurst}{Dr. Wrayno}{Math 301}{\today}{In the Pursuit of the Unknown}
Ever since I was little, I have had an interest in computers. It sounds corny but the earliest memory that I can recall is sitting in front of my grandparents' computer and wondering just how the magical mystery box in front of me was working. When I was ten years old, I started to develop an interest in math as most people were struggling with concepts such as, ``what is $x$,'' I was very quickly able to grasp the concept of a variable. Twelve years later, I am now about to finish my degree as well as start graduate classes for a degree in Computer Science. I am also finishing up my minor in Mathematics which will turn into another bachelors degree during my graduate studies. I have had an interest in computers and mathematics for a while, and the story of how I decided to be a Computer Science and a Mathematics major is an interesting one to say the least.\\

My interest in computers started with my grandparents since they were the first in my immediate family to own a computer. I vividly remember sitting in front of the monitor and thinking that this is a television that I can control. I was instantly hooked and wanted to learn everything about the magical mystery box. With a little bit of research and understanding, I found out that the magical mystery box was not a television, but more of what people think computers are today. A couple years later at a Cub Scout meeting, we took apart a computer. This took away the proverbial curtain and I then learned that a computer was nothing more than transistors and silicon. This drove me to discover even more about how computers worked and functioned. My parents took note of how fascinated I was with computers and sent me to a summer camp where I learned the basics of web programming and programming simple games with Visual Basic. Then, with a passion for computers, I went off to college to learn and absorb as much information as possible. Now I am here at Christopher Newport University getting my degree in Computer Science and currently starting my graduate classes for a masters degree.\\

Mathematics is something that I happened to be good at. I always got the joy and satisfaction from helping my peers, and as the school ``math wiz" I was able to reinforce my mathematics skills. When I was in elementary school, the first time I knew I was good at math was when I took the timed multiplication tests. I was able to complete them pretty quickly and scored top in my class for math. My teacher at the time took note of how quickly I was able to solve these and suggested I play the math card game \textit{24}. \textit{24} was a game with four numbers on a card, the goal of the game was to be the first to use addition, subtraction, multiplication and division to get 24. At my school I was really good at this game. I was so good in fact, that I went to a state wide ``math tournament" to play this game against other schools from other districts. The math from middle and high school came easy to me and supplied no challenges. To supplement this need for a challenge I started to teach myself more advanced math that would not be brought up until I was in college. Now that I am here, I keep challenging myself and try to push the limits of what I know about mathematics. This year I am finishing up my Mathematics minor and will continue to pursue a mathematics major during my graduate studies.\\

Knowledge to me, is something to strive for. Admitting that you may not have all the answers is the first step in obtaining those answers. From a young age I was always curious about how things worked and finding a solution to the unknown. Computer Science helped me pursue the understanding of how things work and Mathematics helped me pursue finding a solution to the unknown. I do not know where the future will take me but I know with computer science and mathematics that I will continue to pursue the unknown in the pursuit of knowledge.
\end {mla}
\end{document}
